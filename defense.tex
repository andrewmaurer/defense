\documentclass{beamer}

\usepackage{macros}

\usetheme{Boadilla}
\usecolortheme{beaver}

\begin{document}

\author[A. Maurer]{Andrew Maurer}
\date{September 23, 2017}
\title[On the FG of $\H^\bullet$ for LSA]{On the Finite Generation of Relative Cohomology for Lie Superalgebras}

\begin{frame}
  \maketitle
\end{frame}

\begin{frame}{Overview}
  \tableofcontents
\end{frame}



\section[LSAs \& $\H^\bullet$]{Lie superalgebras and their cohomology}

\begin{frame}{Lie Superalgebras}
  \pause
  \begin{definition}
    A Lie superalgebra $\g = \even{\g} \oplus \odd{\g}$ is a $\ZZ_2$-graded (complex) vector space with a bilinear bracket $[\cdot,\cdot]: \g \otimes \g \to \g$ satisfying: \pause
    \begin{itemize}
    \item $[x,[y,z]] = [[x,y],z] + (-1)^{\bar x \bar y} [y,[x,z]]$\pause
    \item $[x,y] + (-1)^{\bar x \bar y} [y,x] = 0$
    \end{itemize}
    for homogeneous $x,y,z \in \g$.
  \end{definition}\pause
  \begin{example}
    The degree-zero elements $\even{\g}$ form a Lie algebra.
  \end{example}

\end{frame}

\begin{frame}{Examples of Lie superalgebras}
  \begin{itemize}
    \pause
  \item $\gl(m|n) = $ all $(m+n) \times (m+n)$ matrices.
    \[
\even{\gl(m|n)} = \left(
  \begin{array}{c|c}
    A & 0 \\
    \hline
    0 & D
  \end{array}
\right)
\text{ and } \odd{\gl(m|n)} = \left(
  \begin{array}{c|c}
    0 & B \\
    \hline
    C & 0
  \end{array}
\right)
\]
\pause
  Bracket is given by $[X,Y] = X \circ Y - (-1)^{\bar X \bar Y} Y \circ X$.
  \pause\item $\sl(m|n)$ is the subsuperalgebra of $\gl(m|n)$ given by all matrices of \emph{supertrace} 0, i.e., those such that $\Tr A - \Tr D = 0$.
  \pause\item Those defined similarly to Lie algebras: linear transformations preserving certain forms on $\ZZ_2$-graded vector spaces.
  \end{itemize}
\pause
  The simple, complex, finite-dimensional Lie superalgebras were classified [Kac].
\end{frame}

\begin{frame}{Classical Lie superalgebras}\pause
  \begin{definition}
  A Lie superalgebra $\even{\g} = \even{\g} +\odd{\g}$ is \emph{classical} if there is a reductive algebraic group $\even{G}$ such that
  \begin{itemize}
  \item $\even{\g} = \Lie(\even{G})$, and
  \item An action $\even{G} \actson \odd{\g}$ differentiates to the adjoint action.
  \end{itemize}
  \end{definition}\pause
  \begin{example}
    $\gl(m|n)$, $\sl(m|n)$, $\mathfrak{osp}(m|n)$.
  \end{example}
\end{frame}

\begin{frame}{Modules for Lie superalgebras}\pause
  \begin{definition}
    A \emph{module} for $\g$ is a $\ZZ_2$-graded vector space $M = \even{M} \oplus \odd{M}$ with a homomorphism $\varphi: \g \to \gl(M)$. In this way $\g$ acts on $M$.
  \end{definition}\pause
  \begin{example}
    \begin{itemize}
    \item $\CC$ is the trivial $\g$-module.\pause
    \item $\g$ is a $\g$-module via the adjoint action.
    \end{itemize}
    
  \end{example}\pause
  A $\g$-module is really a graded module for $U_s(\g)$, the universal enveloping superalgebra:
  \[
    U_s(\g) = \mathcal{T}(\g) \bigg/ \left< \lambda \otimes \mu - (-1)^{\bar \mu \cdot \bar \lambda} \mu \otimes \lambda  - [\lambda, \mu]\right>.
  \]
\end{frame}

\begin{frame}{Wedges}\pause
  \begin{definition}
    The \emph{exterior product} of a $\g$-module $M = \even{M} \oplus \odd{M}$ is
    \[
      \superext{p}\left(M\right) = \mathcal{T}\left(M\right)/\left< \lambda \otimes \mu + (-1)^{\bar \mu \cdot \bar \lambda} \mu \otimes \lambda \right>
    \]\pause
    or
    \[
      \superext{p}\left(M\right) = \bigoplus_{i+j=p} \ext{i}\left(\even{M}\right) \otimes {S}^j\left(\odd{M}\right)
    \]\pause
 This is naturally a $\g$-module.   
  \end{definition}
\end{frame}

\begin{frame}{Relative cohomology I}\pause
  Relative cohomology fits into Hochschild's relative cohomology theory. \pause We use an explicit Koszul complex. \pause Let $\a \leq \g$ be a subsuperalgebra:
  \[
    C^p(\g,\a;M) = \Hom_\a\left(\superext{p}\left( \g/\a \right), M \right)
  \]\pause
  Equip this with differentials:
  \[
    d: C^{p}(\g,\a;M) \to C^{p+1}(\g,\a;M)
  \]\pause
  \begin{align*}
    df(\w_0\wedge\ldots\wedge\w_p) = &\sum_i (-1)^{\tau_i(-)} \w_i . f(\w_0 \wedge \ldots \hat \w_i \ldots \wedge \w_p) \\
    &+ \sum_{i < j} (-1)^{\sigma_{i,j}(-)} f([\w_i,\w_j]\wedge \w_0 \ldots \hat \w_i \ldots \hat \w_j \ldots \wedge \w_p)
  \end{align*}
\end{frame}

\begin{frame}{Relative cohomology II}\pause
  \begin{definition}
    \[
      \H^p(\g,\a;M) = \frac{\ker\left(d:C^p(\g,\a;M) \to C^{p+1}(\g,\a;M)\right)}{\im\left(d: C^{p-1}(\g,\a;M) \to C^p(\g,\a;M)\right)}
    \]
    \pause Or
    \[
      \H^p(\g,\a;M) = \Ext^p_{(\g,\a)}(\CC,M)
    \]
  \end{definition}\pause
  \[
    \H^p(\g,\a;M) = \frac{\left\{0 \to M \to E_1 \to \ldots \to E_p \to \CC \to 0 \text{ which are $\a$-split}\right\}}{\sim}
  \]
\end{frame}

\begin{frame}{Relative cohomology III}\pause
  We can give the following direct sum of cohomology groups the structure of a graded $\CC$-algebra by Yoneda splice or tensor product:
  \[
    \H^\bullet(\g,\a;\CC) = \bigoplus_{p \in \ZZ} \H^p(\g,\a;\CC)
  \]\pause
  Under tensor product, the direct sum of cohomology groups for a module becomes a module for $\H^\bullet(\g,\a;\CC)$:
  \[
    \H^\bullet(\g,\a;M) = \bigoplus_{p \in \ZZ} \H^p(\g,\a;M).
  \]\pause
  As does
  \[
    \Ext^\bullet_{(\g,\a)}(M,M) = \bigoplus_{p \in \ZZ} \Ext^p(\g,\a;M).
  \]
  
\end{frame}

\begin{frame}{A result of Fuks-Leites}\pause
  \begin{theorem}[Fuks-Leites]
    There are isomorphisms:
    \[
      \H^\bullet(\mathfrak{osp}(m|2n),0;\CC) \cong
      \begin{cases}
        \H^\bullet(\mathfrak{o}(m),0;\CC) \text{ if } m \geq 2n \\
        \H^\bullet(\mathfrak{sp}(2n),0;\CC) \text{ if } m < 2n
      \end{cases}
    \]
    \[
      \H^\bullet(\gl(m|n),0;\CC) \cong \H^\bullet(\gl(\max(m,n)),0;\CC)
    \]
  \end{theorem}\pause
  There's a similar statement for Lie superalgebras of type $G(3)$, $F(4)$, and $D(2,1;\alpha)$
\end{frame}

\begin{frame}{A result of Boe-Kujawa-Nakano}

  \begin{theorem}[Boe, Kujawa, Nakano 2006]
    Let $\g = \even{\g} \oplus \odd{\g}$ be a classical Lie superalgebra. The relative cohomology ring $\H^\bullet(\g,\even{\g};\CC)$ is isomorphic to $S^\bullet(\odd{\g}^*)^{\even{G}}$, and is thus a finitely generated $\CC$-algebra.
  \end{theorem}
\end{frame}


\section{Spectral Sequence}

\begin{frame}{Spectral sequence theorem}\pause

  \begin{theorem}
    Let $\g = \even{\g} \oplus \odd{\g}$ be a classical Lie superalgebra, $\a \leq \even{\g}$ a subalgebra, and $M$ a $\g$-module. There exists a spectral sequence $\{E_r^{p,q}(M)\}$ which computes cohomology and satisfies
    \begin{align*}
      E_2^{p,q}(M) \cong \H^p(\g,\even{\g};M) \otimes \H^q(\even{\g},\a;\CC) \Rightarrow \H^{p+q}(\g,\a;\CC).
    \end{align*}
    Moreover, when $M$ is finite-dimensional, $E_2^{\bullet,\bullet}(M)$ is a Noetherian $E_2^{\bullet,\bullet}(\CC)$-module.
  \end{theorem}
\pause
  \begin{corollary}
    The relative cohomology ring $\H^\bullet(\g,\a;\CC)$ is a finitely-generated $\CC$-algebra.
  \end{corollary}
\end{frame}

\begin{frame}{Filtration}\pause
  We filter the cochains in a way inspired by Hochschild and Serre:\pause
  \begin{enumerate}
  \item Decompose:
    \[
      C^n(\g,\a;M) = \bigoplus_{i + j = n} C^i\left(\even{\g},\a;\Hom_\CC\left( \superext{j}\left(\g/\even{\g}\right),M\right)\right)
    \]\pause
  \item Filter:
    \[
      C^n(\g,\a;M)_{(p)} = \bigoplus_{\substack{i + j = n\\ i \leq n-p}} C^i\left(\even{\g},\a;\Hom_\CC\left( \superext{j}\left(\g/\even{\g}\right),M\right)\right)
    \]
    
  \end{enumerate}
\pause
  We are guaranteed a spectral sequence of $\a$-modules:
  \[
    E_r^{p,q}(M) \Rightarrow \H^{p+q}(\g,\a;M).
  \]  
\end{frame}

\begin{frame}{ID Pages}
  \begin{enumerate}
  \pause\item $E_0^{p,q}(M) \cong C^q(\even{\g},\a;\Hom_\CC(\superext{p}(\g/\even{\g}),M))$: Just a quotient.
    \vspace{0.2in}
  \pause\item $E_1^{p,q}(M) \cong \H^q(\even{\g},\a;\Hom_\CC(\superext{p}(\g/\even{\g}),M))$: Requires some algebraic gymnastics and a diagram chase.
    \vspace{0.2in}
  \pause\item $E_2^{p,q}(M) \cong \H^p(\g,\even{\g};M) \otimes \H^q(\even{\g},\a;\CC)$: This uses the fact that $\g$ is classical and a vanishing theorem for cohomology.
    
   $\Hom_\CC(\superext{p}(\g/\even{\g}),M) = \Hom_{\even{\g}}(\superext{p}(\g/\even{\g}),M) \oplus V = C^p(\g,\even{\g};M) \oplus V$. 
  \end{enumerate}
\end{frame}

\begin{frame}{ID Edge}
  \pause
  We have an edge homomorphism:
  \[
    {E_2^{\bullet,0}(M)} \to E_\infty^{\bullet,0}(M)
  \]\pause
  But recall $E_2^{\bullet,0}(M) \cong \H^\bullet(\g,\even{\g};M)$.
\pause
  \begin{theorem}
    The edge homomorphism $E_2^{\bullet,0}(M) \to E_\infty^{\bullet,0}(M)$ is induced by the restriction map
    \[
      \resmap:\H^\bullet(\g,\even{\g};M) \to \H^\bullet(\g,\a;M).
    \]
    
  \end{theorem}
\end{frame}

\section{Finite Generation}

\begin{frame}{Finite generation}
  \pause
  \begin{theorem}
    When $M$ is finite-dimensional, $E_r^{\bullet,\bullet}(M)$ is a Noetherian $E_r^{\bullet,\bullet}(\CC)$-module for $2 \leq r \leq \infty$.
  \end{theorem}
  \pause
  \begin{corollary}
    $E_\infty^{\bullet,\bullet}(\CC)$ is a Noetherian $E_2^{\bullet,0}(\CC)$-module, and is thus finitely-generated.
  \end{corollary}
  \pause
  \begin{corollary}
    $\H^\bullet(\g,\a;\CC)$ is Noetherian $\H^\bullet(\g,\even{\g};\CC)$-module, and is therefore finitely-generated as a $\CC$-algebra.
  \end{corollary}
\end{frame}

\begin{frame}{Collapsing at $E_2$}\pause
  \begin{definition}
    Recall that a $\CC$-algebra $A$ is \emph{Cohen-Macaulay} if there is a polynomial subring $R$ such that $A$ is a finite and free $R$-module.
  \end{definition}
  \pause
  \begin{theorem}
   If the spectral sequence collapses at $E_2$, then $\H^\bullet(\g,\a;\CC)$ is a Cohen-Macaulay ring.
  \end{theorem}
 \pause
   This uses a rather serious result of Hochster-Roberts regarding invariants under a reductive group action being Cohen-Macaulay.
\end{frame}

\begin{frame}{Application}\pause
  \begin{theorem}
    Let $\g = \even{\g} \oplus \odd{\g}$ be a classical Lie superalgebra such that $\H^\bullet(\g,\even{\g};\CC)$ vanishes in odd degrees. Let $\l \leq \even{\g}$ be a standard Levi subalgebra (i.e., nonzero and generated by simple roots). The following hold:
    \begin{itemize}
    \pause\item Spectral sequence collapses at $E_2$.
    \pause\item $\H^\bullet(\g,\l;\CC)$ is Cohen-Macaulay.
    \pause\item $\kdim \H^\bullet(\g,\even{\g};\CC) = \kdim \H^\bullet(\g,\l;\CC)$.
    \end{itemize}
  \end{theorem}
  
  \pause This statement uses deep results from Kazhdan-Lusztig theory and Category $\O$ cohomology.

  \pause
\begin{example}
  This result applies to many $\g$. E.g., $\gl(m|n)$, $\sl(m|n)$, $\mathfrak{psl}(2n|2n)$, $\mathfrak{osp}(2m+1|2n)$, $\mathfrak{osp}(2m,2n)$, $P(4\ell - 1)$, $D(2,1;\alpha)$, $G(3)$, and $F(4)$.
\end{example}
\end{frame}

\section{Geometry}

\begin{frame}{Cohomology varieties} \pause
  To do geometry, we need a finitely-generated commutative subring. As in Friedlander-Parshall, we use the even cohomology groups:
  \[
    \H^{ev}(\g,\a;\CC) = \bigoplus_{n \geq 0} \H^{2n}(\g,\a;\CC).
  \]
  \pause
  \begin{definition}
    The \emph{relative cohomology variety} of $\g$ relative to $\a$ is
    \[
      V_{(\g,\a)}(\CC) = \maxspec\left(\H^{ev}(\g,\a;\CC)\right).
    \]
    
  \end{definition}
\end{frame}

\begin{frame}{Support varieties}\pause
  A $\g$-module $M$ determines a $\H^\bullet(\g,\a;M)$-module $\Ext^\bullet_{(\g,\a)}(M,M)$.\pause This determines a variety
  \[
    \operatorname{Z}\left(\Ann_{\H^\bullet(\g,\a;\CC)}\Ext^\bullet_{(\g,\a)}(M,M)\right) \subseteq V_{(\g,\a)}(\CC).
  \]\pause
  Call this the \emph{support variety} $V_{(\g,\a)}(M) \subseteq V_{(\g,\a)}(\CC)$.
\end{frame}

\begin{frame}{Realization morphism and Natural Modules}\pause
  We have the restriction map
  \[
    \H^\bullet(\g,\even{\g};\CC) \xrightarrow{\resmap} \H^\bullet(\g,\a;\CC)
  \]\pause
  And a corresponding \emph{realization morphism}
  \[
    V_{(\g,\even{\g})}(\CC) \xleftarrow{\Phi} V_{(\g,\a)}(\CC)
  \]
\pause
  \begin{definition}
    We say a $\g$-module is \emph{natural} relative to $\a$ if
    \[
      \Phi(V_{(\g,\a)}(M)) = \Phi(V_{(\g,\a)}(\CC)) \cap V_{(\g,\even{\g})}(M).
    \]
    The subalgebra $\a$ is \emph{natural in $\g$} if every $\g$-module is natural relative to $\a$.
  \end{definition}
  
\end{frame}

\begin{frame}{Tensor products}\pause
  We say a Lie superalgebra $\g$ \emph{satisfies the tensor product theorem} relative to $\a$ if
  \[
    V_{(\g,\a)}(M \otimes N) = V_{(\g,\a)}(M) \cap V_{(\g,\a)}(N)
  \]
  for all modules $M$ and $N$.
  \pause
  \begin{theorem}
    Let $\g$ be a Lie superalgebra which satisfies the tensor product theorem relative to $\even{\g}$. Then \[\Phi(V_{(\g,\a)}(M \otimes N)) = \Phi(V_{(\g,\a)}(M)) \cap \Phi(V_{(\g,\a)}(N)).\]
  \end{theorem}
\end{frame}

\begin{frame}{Connectedness} \pause
  We need to assume further $\g$ is \emph{stable} and \emph{polar}, some GIT conditions appearing in [Bagci-Kujawa-Nakano].

  \pause
  \begin{theorem}
    Let $\g = \even{\g} \oplus \odd{\g}$ be classical, stable, and polar, with $\a \leq \even{\g}$ natural. Suppose $\Phi(V_{(\g,\a)}(M)) = X \cup Y$ with $X \cap Y = \{0\}$. Then there exist modules $M_1$ and $M_2$ with
    \[
      X = \Phi(V_{(\g,\a)}(M_1)) \text{ and } Y = \Phi(V_{(\g,\a)})(M_1).
    \]
    
  \end{theorem}
\end{frame}

\begin{frame}{Realizability}
  \pause
  \begin{theorem}
    Let $\g = \even{\g} \oplus \odd{\g}$ be classical, stable, and polar, with $\a \leq \even{\g}$ natural. Let $X$ be a closed, conical, subvariety of $V_{(\g,\a)}(\CC)$. 
  \end{theorem}
\end{frame}

\begin{frame}{What's next?}
  \begin{enumerate}
  \pause\item What happens when $\a \not\leq \even{\g}$? Are there cases when $\H^\bullet(\g,\a;\CC)$ is still finitely-generated?
  \pause\item We saw that the spectral sequence collapses at $E_2$ implies $\H^\bullet(\g,\a;\CC)$ is CM. What about the converse? (It does for $p$-nilpotent Lie algebras, [Carlson-Nakano].)
  \pause\item When is $\a$ natural?
  \pause\item The dimension $\dim V_{(\g,\even{\g}}(M) = \atyp M$ for simple modules when $\g = \gl(m|n)$ [Boe-Kujawa-Nakano]. Is there a combinatorial interpretation for $\dim V_{(\g,\a)} M$ for simple $M$?
  \end{enumerate}
\end{frame}

\begin{frame}{Thanks!}
  \centering
  \large Thank you!
\end{frame}

\end{document}
